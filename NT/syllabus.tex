\documentclass[a4, 12pt]{article}
\usepackage{hyperref}
\usepackage[right=2.5cm, left=2.5cm,top=1.5cm]{geometry}
\usepackage[french]{babel}
\usepackage[style=alphabetic,sorting=nyt,sortcites=true,autopunct=true,babel=hyphen,hyperref=true,abbreviate=false,backref=true,backend=biber]{biblatex}

%%%%%% FONTSTYLE
\usepackage[heuristica,vvarbb,bigdelims]{newtxmath}
\usepackage{fontspec}
\setmainfont{Eczar-Regular.otf}[BoldFont=Eczar-SemiBold.otf]
\renewcommand*\oldstylenums[1]{\textosf{#1}}
%%%%%%%%%%%%

%%% ENVIRONMENT BOXES
\usepackage[most]{tcolorbox}
\newtcolorbox{skills}
{
colback=teal!5,
frame hidden,
sharp corners,
enhanced,
borderline west={3pt}{0pt}{teal}}
\newtcolorbox{info}
{
colback=pink!5,
frame hidden,
sharp corners,
enhanced,
borderline west={3pt}{0pt}{pink},
}
%%%%%%%%%%%%
\providecommand{\keywords}[1]{\textbf{\textit{Définitions à retenir---}} #1}


%%%% BIBLIOGRAPHY
\addbibresource{bibliography.bib}
\defbibheading{bibempty}{}
%%%%%%%%%%%%


\title{Syllabus \\
Introduction à la critique textuelle néo-testamentaire\vspace{-.5cm}}
\date{Institut Protestant de Théologie - Site de Montpellier}
\author{Sophie Robert-Hayek, Université de Lorraine}

\begin{document}

\maketitle

% \begin{skills}{}
% Test
% \end{skills}

% \section{Organisation des séances}

% \subsection{Qu'est-ce que la critique textuelle ?}

% \begin{itemize}
%     \item Définitions;
%     \item Rôle et objectifs;
%     \item Bref historique;
%     \item Présentation de nos outils de travail:
%         \begin{itemize}
%             \item Le Nestlé-Aland;
%             \item Le \emph{New Testament Virtual Manuscript Room} (NTVMR);
%             \item StepBible (\verb|stepbible.org|)
%         \end{itemize}
% \end{itemize}

% \subsection{Paléographie et codicologie des manuscrits néo-testamentaires}
% \begin{itemize}
%     \item Les langues du Nouveau Testament;
%     \item Codicologie des manuscrits;
%     \item Paléographie des manuscrits;
%     \item Impact des annotations sur la critique textuelle;
%     \item Cas d'étude: \textbf{le Sinaïticus}.
% \end{itemize}

% \subsection{Classifier les manuscrits néo-testamentaires}

% \begin{itemize}
%     \item Les classifications paléographiques/
%     \item Les classifications sur la \og qualité \fg\ du texte;
%     \item Les classifications sur le contenu du texte:
%         \begin{itemize}
%             \item Classification de Von Soden;
%             \item Classification de Wisse;
%             \item Classification de Colwell;
%         \end{itemize}
% \end{itemize}

% \subsection{Éditer le texte néo-testamentaire}

% \begin{itemize}
%     \item Présentation de la méthode;
%     \item Présentation et étude d'une édition critique majeure: le Nestlé-Aland;
% \end{itemize}

% \subsection{Séance de travaux pratique}
% \begin{itemize}
%     \item Présentation de plusieurs manuscrits majeurs;
%     \item Réalisation à la main de l'édition critique de Marc 1,1-2;
%     \item Comparaison avec le Nestlé-Aland et discussion.
% \end{itemize}

% \subsection{Nouvelles directions en critique textuelle}
% \begin{itemize}
%     \item Les nouveaux outils d'édition (collation automatique, XML TEI, \dots);
%     \item Les approches quantitatives appliquées au texte biblique.
% \end{itemize}

\section{Plan du cours}
\begin{enumerate}
    \item Introduction, définitions, et objectifs;
    \item Les sources du texte néo-testamentaire;
    \item Les premiers textes imprimés;
    \item Les principes de la critique textuelle : cas pratique(s);
    \item La critique textuelle à l'ère du numérique.
\end{enumerate}

\section{Modalités de validation}

La validation du cours s'effectue par la préparation d'un dossier (> 5 pages), qui présentera pour Marc 1,1-2 et pour les témoins proposés dans le cas pratique :
\begin{itemize}
    \item La collation des textes;
    \item L'examen des variants;
    \item La critique interne des leçons;
    \item La critique externe des manuscrits;
    \item L'établissement du texte;
    \item La comparaison par rapport au texte retenu par le Nestlé-Aland.
\end{itemize}

\section{Bibliographie conseillée}

\paragraph{Ouvrages généraux}

\begin{itemize}
    \item \fullcite{West1973}
\end{itemize}

\paragraph{Éditions du Nouveau Testament en Grec}
\begin{itemize}
    \item \fullcite{NestleAland2012}
    \item \fullcite{WestcottHort1881}
    \item \fullcite{Tischendorf1864}
    \item \fullcite{Stephanus1550}
    \item \fullcite{Beza1598}
    \end{itemize}

\paragraph{Critique textuelle néo-testamentaire}



\begin{itemize}
    \item \fullcite{aland1995text}
    \item \fullcite{greenlee1964introduction}
    \item \fullcite{Vaganay1986}

\end{itemize}

\paragraph{Paléographie grecque}

\begin{itemize}
    \item \fullcite{metzger1981manuscripts}
    \item \fullcite{aland1995text}
    \item \fullcite{vangroningen1967short}
    \item \fullcite{cavallo1967ricerche}
    \item \fullcite{janz_greek_paleography}
    \item \fullcite{bausi2015comparative}
\end{itemize}

\paragraph{Outils divers}
\begin{itemize}
    \item \url{https://ntvmr.uni-muenster.de} (New Testament Virtual Manuscript Room)
    \item \url{https://digi.vatlib.it} (Digital Vatican Library)
    \item \url{https://archivesetmanuscrits.bnf.fr}(Manuscrits numérisés BNF)
\end{itemize}
\end{document}