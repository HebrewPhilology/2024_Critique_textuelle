Définition de la critique textuelle (selon Henlee):

\textit{La critique textuelle est l'étude des copies de toute œuvre écrite dont l'autographe (l'original) est inconnu, dans le but de déterminer le texte original.}

-> Donc absolument pas limité à la recherche en science biblique, mais s'applique à n'importe quel texte écrit et diffusé avant l'invention de l'imprimerie (car lorsque la copie est faîte à la main, aucune garantie qu'on ait deux fois le même texte)

-> On a un ensemble de variants et d'erreurs qui se propagent depuis chaque variante du texte.

-> à ne pas confondre avec la critique des sources par exemple, qui cherche un ensemble de sources qui seraient sous jacentes au texte néo-testamentaire;


-> Donc on est pas dans une recherche qui serait unique au texte néo-testamentaire, même si certaines de ces particularités fait qu'il faut une certaine adaptation.

-> Particularités du Nouveau Testament:
-> Il s'agit d'une des oeuvres de la littérature occidentale les plus importantes;
-> Il s'agit de la plus grande oeuvre existante en terme de volume de manuscrits retrouvés, avec plus de X manuscrits et un ensemble de Y variants qui ont été retrouvés.
-> Même si on a l'impression qu'il y a une grosse distance, il s'agit d'un des textes avec la moindre distance entre ce qui serait une version initiale et les premières copies dont nous pouvons avoir accès.


-> On peut donner des points de comparaison avec d'autres oeuvres existantes:

the plays of Aeschylus are known in some fifty Mss, the works
of Sophocles in one hundred, the Greek Anthology and the An-
nals of Tacitus in one MS each, the poems of Catullus in three
Mss of independent value, and there are a few hundred known mss of works of Euripides, Cicero, Ovid, and Virgil. In the case
of the NT, in sharp contrast, there are some 5000 extant MSS in
Greek,2 8000 in Latin, and 1000 in other languages. As regards
the time interval between the extant MSs and the autograph,
the oldest known ss of most of the Greek classical authors are
dated a thousand years or more after the author's death. The
time interval for the Latin authors is somewhat less, varying
down to a minimum of three centuries in the case of Virgil. In
the case of the NT, however, two of the most important MSS
were written within 300 years after the NT was completed, and
some virtually complete NT books as well as extensive fragmen-
tary Mss of many parts of the NT date back to one century from
the original writings.
(with the particularity that the classics copy seem to have overall less variants.)


-> Est-ce qu'on pose la question de la stemmatologie pour retrouver l'original ? Dans un mode idéal, on pourrait avoir tous les manuscrits et en faire un stemma et prendre le plus vieux. Mais impossible et on choisit un modèle eccléctique dans la recherche moderne (avec des éditions diplomatiques du Sinaïticus tel qu'édité par Tisch et compagnie)

-> Permet l'établissement d'un texte de travail donc (New Testament textual criticism, therefore, is
the basic biblical study, a prerequisite to all other biblical and
theological work.), tout en permettant de prendre en compte dans toute anlayse théologique les possibles variations (surtout à certains points du texte qui sont plus compliqués).\\


15 nomina sacra (NOMEN SACRUM).

hus, in marked distinction
from other abbreviations, these contractions were not made for
the purpose of saving space or labor. That this is true is seen in
two related facts. In the first place, contraction as a type of
abbreviation is distinctly limited to the Mss of the Bible and of
Christian literature, and is virtually unknown in secular litera-
ture. In the second place, even in biblical Mss these very same
words are often not contracted if they are used in any other
than the specialized sense; e.g., tatnp is usually contracted only
when it refers to God, and Gv@paroc only in such references as
“the Son of Man’ as a title of Jesus. 

Suspension. Suspension is the ordinary type of ab-
breviation, used to save time or space, and used especially at
the end of a line. Suspension is indicated in one of several
ways:
(i) The first letter only may be written, with a charac-
teristic mark to show suspension: e.g., 0 (vidc), «9 (Kat).
(ii) The first part of the word may be written, with a hori-
zontal line above the last letter to indicate suspension: e.g., TEX
(téX.06¢).
(iii) In NT uncial Mss, suspension is confined almost entirely
to the omission of a final v at the end of a line, indicated by a
horizontal line above and following the last written letter: e.g.,
TMOAL (MOAtv).
(iv) The first part of the word may be written with the last
written letter or letters above the line and smaller: e.g., te*
(téA0c), TE (TEKVA).

La question des suspensions est pas mal (p. 21 du texte de Greenle)

Tableau sympa p.23.

Voir pour les abbréviations les plus communes dans: Thompson, Palaeography, pp.
80-84, voir si jamais on peut en faire la liste.\\

On a 98 papyri du NT, qui vont de livres entier à du fragmentaire.

Of course, when these Mss are housed in a library
they usually have a local library catalog number as well.\\


Utility of the studying the versions in different language -> the form of the Greek text which was in use at that period and
in that geographical region.

But take into account the particularities of the source and the target language.

Old Latin (Itala)
-> the Itala seems to have
been copied and used to some extent until the ninth century
or later.

Finally in
382 Pope Damasus commissioned Jerome, his advisor and the
outstanding biblical scholar of his day, to undertake a revision
of the Latin Bible, standardizing it by the “true Greek text.” 

No less than 8000 Mss of the Vulgate are now known, or many
more than all the known Greek NT Mss. This suggests that the
Vulgate Bible was the most frequently copied book of all an-
cient literature. 


The Peshitta contained all of the NT books except 2 Peter,
2 and 3 John, Jude, and Revelation. These were omitted because
they were not recognized as canonical by the Syrian church.3
The Peshitta is known in 350 Mss or more, the oldest dating
from the fifth century.


Until recently, the earliest of
the approximately 100 known Bohairic NT Mss were from the
ninth century and later, and some scholars had suggested that
the Bohairic NT originated no earlier than the seventh or eighth
century. Yet it would seem strange if the dialect of Alexandria
was without a NT for centuries after other dialects had the
Scriptures. 


The NT seems to have been translated into the Armenian
language in the early part of the fifth century under the spon-
sorship of the Patriarch Sahak, with most of the actual work
being done by a monk Mesrop, who likewise was responsible
for the Armenian alphabet.

The Christian message was known in Georgia, the moun-
tainous area between the Black and Caspian Seas, in the fourth
century. The NT was translated into Georgian by about the middle
of the fifth century, using an alphabet which tradition credits to
the same Mesrop who is said to have developed the Armenian
alphabet. 


This makes it difficult to answer
questions concerning the origin of the Ethiopic NT. Scholars
have assigned it to dates as early as the second century and as
late as the fourteenth. The version has been said to follow the
Greek text slavishly at times, and it has been said to have been
translated from the Syriac.

The third principal source of knowledge of the NT text is
the great number of quotations from the NT which are found
in the writings of Christian writers of the early centuries. These
quotations are so extensive that the NT could virtually be
reconstructed from them without the use of NT Mss. Most
of the quotations are found in Greek and Latin documents,
with an appreciable amount in Syriac and some in a few other
languages.

p. 46 -> Cool questions regarding the patristic quotations.

These
quotations are so extensive that the NT could virtually be
reconstructed from them without the use of NT Mss. \\

Histoire de la transmission du texte et des différentes phases de transmission de celui-ci.

(period of divergence of manuscripts, until 325)

When Christianity attained official status under Constantine,
mss of the NT needed no longer be kept concealed for safety. Very
soon afterwards the emperor himself ordered fifty new copies of
the Bible to be made for the churches of Constantinople.

8th century -> Becomes the reign of the standardized text.

The printed Greek NT was likewise basi-
cally a Byzantine type of text and continued to be so until the
latter part of the nineteenth century.\\


-> Question de la classification des variants et de leur impact sur les lectures de critique textuelle.

Question de ce qui se passe pour le texte du Nouveau Testament une fois que l'impression est inventée.

For the first time it became possible to reproduce a docu-
ment in an unlimited number of copies, and to have these copies
absolutely identical in their text. The difference this invention
made for the civilized world is almost beyond comprehension.


Yet the printing press signaled the be-
ginning of a new age in which literature would no longer de-
pend upon single copies tediously made by hand, and when
books could be owned by the masses instead of the wealthy few
alone.

On a l'impression de la Bible de Gutenberg (en Latin), puis la naissance par un évêque espagnol de la Polyglotte de Complutum.

Puis Erasme en 1616 (selon plusieurs éditions, de qualité variable, avec seulement Codex 1 qui n'est pas un byzantin médiéval).\\

Puis Stephanus, édition regia, qui est le texte officiel "receptus" et qui a pris une valeur théologique énorme chez la théologie évangélique américaine notamment. On utilise ce terme pour Elzevir et pour la troisième édition de Stephanus (“Textum ergo habes, nunc ab omnibus receptum:
in quo nihil immutatum aut corruptum damus” )

Puis Bezae qui propose son propre texte de manière successive (7 éditions en grec).

Yet three centuries were to pass before scholars won the
struggle to replace this hastily assembled text with a text which
gave evidence of being closer to the NT autographs.\\


Puis \\

(1633-1830) -> On accumule l'évidence propre aux MS.

John Mill, building upon foundations laid by the pre-
vious work of several other scholars, published a large edition
of the Greek NT in 1707. His text was that of the third edition of
Stephanus with only a few changes, and with a critical appara-
tus of the readings of seventy-eight Mss, several versions, and
some patristic writers. Probably no one contributed more in-
formation from these sources of the text for the next century
than did Mill. 



Johann Jakob Griesbach, one of the most important of textual
scholars, who published three editions of the NT between 1774
and 1806 and also collated a large number of Mss. Griesbach
proposed three families of witnesses in the Gospels: (1) Alexan-
drian, including Mss C K L 1 13 33 69, the Bohairic and some
other versions, and the quotations of Origen, Eusebius, and some
other Fathers; (2) Western, including Codex D and the Latin
versions and Fathers; and (3) Byzantine, including A and the
later uncials and most minuscules, which he considered infe-
rior to the other two families.

Lachmann was a classicist and not a theologian; conse-
quently, he was unaware of how violent the criticism against
his work might be. He set the TR aside completely and con-
structed a text from what he believed were the most ancient
witnesses. His first edition was published in 1831 with no ex-
planation of how he had arrived at his text, but merely a state-
ment that a certain theological journal contained an article in
which he had set forth his principles. 

Tischendorf: ltogether he published the
texts of twenty-one uncial Mss of various length and collated or
copied the texts of more than twenty others.

His “eighth
major edition” (1869-72) contains a critical apparatus which
has never been equaled in comprehensiveness of citation of
Greek Mss, versions, and patristic evidence. \\

Brooke Foss Westcott (later bishop of Durham) and Fenton John
Anthony Hort, professor of divinity. For twenty-eight years they
worked together on a critical edition of the Greek NT together
with a volume setting forth with meticulous care their textual
principles. These two volumes were published in 1881—82 under
the title, The New Testament in the Original Greek.

The age of the critical text post Wescott et Hort:

Introduire un ensemble de règle (là on peut les reprendre aux Aland) pour faire les choix dans les variants et pour reconstruire le texte en fonction.


The steps of the procedure, then, are as follows: (1) study
individual readings on the basis of intrinsic probability; (2) evalu-
ate the individual witnesses; (3) determine the family group-
ings of the witnesses; and (4) return to the individual readings
to confirm or revise conclusions.
The conclusions from intrinsic probability are called “inter-
nal evidence.” The testimony of MSS, versions, etc. is known as
“external evidence.”


