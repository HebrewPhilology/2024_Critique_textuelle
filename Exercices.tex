\documentclass[11pt,a4paper]{article}
%%%%%%%%%%%%%%%%%%
% pandoc command %
%%%%%%%%%%%%%%%%%%

%pandoc --bibliography=references.bib --bibliography=references_local.bib -o main.docx main.tex --citeproc --csl=SBL.csl
%I took the .csl file directly from zotero -> modify style -> save as...


\usepackage{hyphenat}
\usepackage{tabularx}
\usepackage{hyperref}
\usepackage{makecell}
\usepackage[utf8]{inputenc}
\usepackage{tipa} %phonetic characters
\usepackage{lua-ul} %pour barrer du texte avec la commande \strikeThrough
\usepackage[autostyle]{csquotes}
\MakeOuterQuote{"}

\usepackage[document]{ragged2e} %Permet de régler les alignements
\let\oldfootnote\footnote
\renewcommand{\footnote}[1]{\oldfootnote{\justifying #1}} %Justify footnotes

\usepackage[dvipsnames]{xcolor} %Allow colors

% -------------------------------
%  Language Hebrew & Greek     
% -------------------------------
\usepackage{polyglossia} %[babelshorthands] permet d'avoir les guillemets allemands avec le code "`toto"' et les guillemets français avec le code "<tata">
\setdefaultlanguage[variant=american]{english}
%\PolyglossiaSetup{english}{indentfirst=false}
\setotherlanguages{french, hebrew, syriac, greek}
\hyphenation{manu-script, manu-scripts, stem-ma, inter-relation-ship}
% ------------------------
% Define fonts
% ------------------------

\usepackage{fontspec}
\setmainfont[]{cochineal}
% %\setmainfont[Path=./fonts/,
%  BoldFont={Brill-Bold.ttf}, 
%  ItalicFont={Brill-Italic.ttf},
%  BoldItalicFont={Brill-BoldItalic.ttf}
%  ]{Brill-Roman.ttf}
 
\newfontfamily{\hebrewfont}[Script=Hebrew, Path=./fonts/]{SBL_Hbrw.ttf}
\newfontfamily{\hebrewfontsf}[Script=Hebrew]{Miriam CLM}
\newfontfamily{\hebrewfonttt}[Script=Hebrew]{Miriam Mono CLM}
%\newfontfamily\hebrewfont[Script=Hebrew]{SBL_BLit.ttf}
\newfontfamily\syriacfont[Script=Syriac, Path=./fonts/]{EstrangeloEdessa.ttf}



% ---------------------------
% Image Package
% ---------------------------
\usepackage{graphicx}
\graphicspath{ {./img/} }
\usepackage{float} %Pour placer les figures au bon endroit
% ---------------------------



% ---------------------------
% Define Bibliography
% ---------------------------
% Load main style
% Pass indexing=cite to biblatex (if you want author indexing)
\PassOptionsToPackage{indexing=cite}{biblatex}
% Pass jblstyle to sbl-paper if you want double spaced footnotes
% \usepackage{sbl-paper}
%\usepackage[style=sbl,doi=false,issn=false,isbn=false,url=false,eprint=false,]{biblatex}
\usepackage[style=sbl]{biblatex}


\DeclareSourcemap{
  \maps[datatype=bibtex]{
    \map{
      \step[fieldset=doi, null]
      \step[fieldset=language, null]
      \step[fieldset=issn, null]{}
      \step[fieldset=url, null]{}
      \step[fieldset=isbn, null]{}
      \step[fieldset=eprint, null]{}
    }
  }
}

%doi=false,issn=false,isbn=false,url=false,eprint=false,

% Add your bib resource here

\addbibresource{references.bib}
\addbibresource{references_local.bib}
%\addbibresource{biblio.bib}
\useshorttitle
\AtEveryBibitem{\clearlist{language}\clearfield{doi}}



%%%%%%%%%%%%%%%%%%%
% Create todolist %
%%%%%%%%%%%%%%%%%%%

\usepackage{enumitem,amssymb}
\newlist{todolist}{itemize}{2}
\setlist[todolist]{label=$\square$}


% ---------------------------------------------------------------------


\title{Introduction à la critique textuelle\\Critique textuelle de la bible hébraïque}
\author{Frédérique Michèle Rey, Sophie Robert-Hayek}
\date{September 2024}

\begin{document}

\maketitle
\justifying

\section{Exercice}
Comprendre et utiliser l'aparat critique de la Biblia Hebraïca Stuttgartensia
\begin{itemize}
    \item Repérer le texte édité
    \item Repérer l'apparat critique
    \item Repérer la Massorah parva
    \item Repérer la Massorah magna
    \item identifier le premier lieu variant de la page
    \item traduire la note d'apparat pour le premier lieu variant
\end{itemize}
\section{Exercice}
Dans les deux examples ci-dessous, en vous basant sur la présentation du texte et sur l'apparat critique, indentifiez quelle édition est eccléctique et laquelle ne l'est pas (inutile de traduire l'hébreu à ce stade.)



\section*{Exercices de critique textuelle}

\subsection*{Exercice 1 – Jg 20:13}
TM: \texthebrew{וְלֹ֤א אָבוּ֙ בִּנְיָמִ֔ן} \\
\texthebrew{אָבוּ} Qal acc. 3ème pers. plur. \texthebrew{אבה} «vouloir» \\
«Ils ne voulurent pas Benjamin» \\
LXX: \textgreek{καὶ οὐκ εὐδόκησαν οἱ υἱοὶ Βενιαμιν} \\
\textbf{εὐδόκησαν} – Aor. 3ème pers. plur. de \textgreek{εὐδοκέω} «vouloir» \\
«Les fils de Benjamin ne voulurent pas»

\subsection*{Exercice 2 – Dt 30:1}
TM: \texthebrew{וַיֵּ֖לֶךְ מֹשֶׁ֑ה וַיְדַבֵּ֛ר אֶת־הַדְּבָרִ֥ים הָאֵ֖לֶּה} \\
4QDeutb: \texthebrew{וַיְכַל משה לְדַבֵּר את כל הד[ברים} \\
\textbf{וַיְכַל} verbe inac. 3ème pers. masc. sing. de \texthebrew{כלה} «finir achever» \\
LXX: \textgreek{καὶ συνετέλεσεν Μωυσῆς λαλῶν πάντας τοὺς λόγους τούτους} \\
\textbf{συνετέλεσεν} - aor. 3 pers. sing. de \textgreek{συντελέω} «finir» \\
\textbf{λαλῶν} - \textgreek{λαλέω} part. pres. nom. masc. sing.

\subsection*{Exercice 3 – Os 6:5}
TM: \texthebrew{וּמִשְׁפָּטֶ֖יךָ אֹ֥ור יֵצֵֽא} \\
\textbf{וּמִשְׁפָּטֶ֖יךָ} - Plur. + suf. 2ème pers. masc. sing. de \texthebrew{מִשְׁפָּט} «tes jugements» \\
LXX: \textgreek{καὶ τὸ κρίμα μου ὡς φῶς ἐξελεύσεται} \\
\textbf{ἐξελεύσεται} - \textgreek{ἐξέρχομαι} ind. fut. moyen 3ème pers. sing. «sortir»

\subsection*{Exercice 4 – Psa 22:30}
TM: \texthebrew{וְ֝נַפְשֹׁ֗ו לֹ֣א חִיָּֽה} \\
\textbf{חִיָּֽה} Piel acc. 3 pers. masc. sing. de \texthebrew{חיה} «vivre» \\
LXX: \textgreek{καὶ ἡ ψυχή μου αὐτῷ ζῇ} \\
\textbf{ζῇ} ind. prés. act. 3 pers. sing. de \textgreek{ζάω} "vivre"

\subsection*{Exercice 5 – Dt 32:8}
TM: \texthebrew{יַצֵּב֙ גְּבֻלֹ֣ת עַמִּ֔ים לְמִסְפַּ֖ר בְּנֵ֥י יִשְׂרָאֵֽל} \\
\textbf{יצב} Verbe Hiph. 3 masc. sing. de \texthebrew{נצב} «placer établir définir» \\
\textbf{גְּבֻלֹ֣ת} «frontières» \\
\textbf{מִסְפַּ֖ר} «nombre» \\
4QDeutj: \texthebrew{בני אלוהים} \\
LXX: \textgreek{ἔστησεν ὅρια ἐθνῶν κατὰ ἀριθμὸν ἀγγέλων θεοῦ} \\
\textbf{ἔστησεν} - \textgreek{ἵστημι} ind. aor. 3ème pers. sing. «établir placer» \\
\textbf{ὅρια} - acc. neutre plur. de \textgreek{ὅριον} «territoire» \\
Les manuscrits de la LXX varient entre \textgreek{ἀγγέλων θεοῦ} et \textgreek{υἱῶν θεοῦ} (voir l’apparat critique de la LXX)

\subsection*{Exercice 6 – Si 41:5}
Ms A: \texthebrew{שׁוֹמֵעַ לִי יִשׁפוֹט אֶמֶת} \\
«Celui qui m’écoute jugera (en) vérité» \\
LXX: \textgreek{ὁ ὑπακούων αὐτῆς κρινεῖ ἔθνη} \\
«Celui qui m’écoute jugera les nations» \\
\textbf{ὑπακούων} - part. prés. de \textgreek{ὑπακούω} «écouter»

\subsection*{Exercice 7 – Gn 15:11}
TM: \texthebrew{וַיַּשֵּׁ֥ב אֹתָ֖ם אַבְרָֽם} \\
\textbf{וַיַּשֵּׁ֥ב} - Hiph. 3 pers. masc. sing. de \texthebrew{נשב} «chasser renvoyer» \\
LXX: \textgreek{καὶ συνεκάθισεν αὐτοῖς Αβραμ} \\
\textbf{συνεκάθισεν} - ind. aor. 3 pers. sing. \textgreek{συγκαθίζω} «s’assoir avec» \\
Indice: comment dit-on «s’assoir habiter demeurer» en hébreu?



\end{document}